\section{Arbeitsmarkt}
\begin{itemize}
	\item Arbeitslosigkeit existiert dann, wenn Haushalte bereit sind zum Marktlohn Arbeit anzubieten, diese aber nicht vom Unternehmen nachgefragt werden.
	\item Keine Arbeitslosigkeit herrscht dann, wenn Haushalte den Marktlohn als zu tief empfinden und deshalb keine Arbeit anbieten.
\end{itemize}

\begin{minipage}{12cm}
    Zwei Typen der Arbeitslosigkeit:
	\begin{itemize}
		\item \textbf{Sockelarbeitslosigkeit (X)}\newline
		 Anzahl der offenen Stellen ist gleich oder grösser als die Anzahl der Arbeitslosen (\textbf{strukturelle und friktionelle} Arbeitslosigkeit)
		\item \textbf{Konjukturelle Arbeitslosigkeit (Y)}\newline
		Anzahl der Arbeitslosen ist grösser als Anzahl der offenen Stellen
	\end{itemize}
\vspace{0.5cm}
Konjunkturelle Arbeitslosigkeit ensteht, wenn das BIP unterhalb der Kapazitätsgrenze liegt.
\end{minipage}
\begin{minipage}{5cm}
	\includegraphics[width=4cm]{images/beveridge.jpg}
\end{minipage}
\subsection{Messung der Arbeitslosigkeit}
\includegraphics[width=0.8\linewidth]{images/messung.png}\\
Arbeitsproduktivität = Menge der produzierten Güter und Dienstleistungen pro Arbeitsstunde
\subsection{Der flexible Arbeitsmarkt}
\begin{multicols}{2}
	\includegraphics[width=\linewidth]{images/felxibellohne.png}
	\includegraphics[width=\linewidth]{images/flexibellohne2.png}
\end{multicols}
\subsection{Der regulierte Arbeitsmarkt}
	\includegraphics[width=0.8\linewidth]{images/fixelohne.png}

	\subsection{Sockelarbeitslosigkeit}
	\begin{itemize}
		\item \textbf{Strukturelle}
		\subitem Mindestlöhne (Fixe Löhne)
		\subitem Zentralisierte Lohnverhandlungen
		\subitem Regulierungen der Anstellung/Entlassung
		\subitem Regulierungen der Arbeitszeit
		\subitem Ausgestaltung der Arbeitslosenversicherung (Bezugshöhe)
		\item \textbf{Friktionelle}
		\subitem Ausgestaltung Arbeitslosigkeit (Bezugsdauer)
		\subitem Zeitspanne zum Finden einer neuen Stelle
	\end{itemize}

	\subsection{Arbeitsmarktpolitik der Schweiz}
	Die Schweiz hat eine sehr kleine Regulierungsdichte, Frankreich hat eine sehr hohe Regulierungsdichte und dadurch eine hohe Arbeitslosigkeit. Freie Arbeitsmärkte erholen sich sehr schnell. 
	\begin{enumerate}
		\item Mindestlöhne
		\subitem CH: Keine branchenübergreifende Mindestlöhne
		\item Zentralisierte Lohnverhandlungen
		\subitem CH: nur dezentral auf Branchenebene
		\item Regulierungen bezüglich Anstellungen und Entlassunge
		\subitem CH: wenig
		\item Ausgestaltung der Arbeitslosenversicherung
		\subitem CH: setzt auf Wiedereingliederung
		\item Regulierung der Arbeitszeit
		\subitem CH: wenig
	\end{enumerate}
\clearpage
\pagebreak


